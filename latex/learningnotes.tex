%% Generated by Sphinx.
\def\sphinxdocclass{report}
\documentclass[letterpaper,10pt,english]{sphinxmanual}
\ifdefined\pdfpxdimen
   \let\sphinxpxdimen\pdfpxdimen\else\newdimen\sphinxpxdimen
\fi \sphinxpxdimen=.75bp\relax
%% turn off hyperref patch of \index as sphinx.xdy xindy module takes care of
%% suitable \hyperpage mark-up, working around hyperref-xindy incompatibility
\PassOptionsToPackage{hyperindex=false}{hyperref}

\PassOptionsToPackage{warn}{textcomp}

\catcode`^^^^00a0\active\protected\def^^^^00a0{\leavevmode\nobreak\ }
\usepackage{cmap}
\usepackage{xeCJK}
\usepackage{amsmath,amssymb,amstext}
\usepackage{polyglossia}
\setmainlanguage{english}



\setmainfont{FreeSerif}[
  Extension      = .otf,
  UprightFont    = *,
  ItalicFont     = *Italic,
  BoldFont       = *Bold,
  BoldItalicFont = *BoldItalic
]
\setsansfont{FreeSans}[
  Extension      = .otf,
  UprightFont    = *,
  ItalicFont     = *Oblique,
  BoldFont       = *Bold,
  BoldItalicFont = *BoldOblique,
]
\setmonofont{FreeMono}[
  Extension      = .otf,
  UprightFont    = *,
  ItalicFont     = *Oblique,
  BoldFont       = *Bold,
  BoldItalicFont = *BoldOblique,
]


\usepackage[Sonny]{fncychap}
\ChNameVar{\Large\normalfont\sffamily}
\ChTitleVar{\Large\normalfont\sffamily}
\usepackage{sphinx}

\fvset{fontsize=\small}
\usepackage{geometry}


% Include hyperref last.
\usepackage{hyperref}
% Fix anchor placement for figures with captions.
\usepackage{hypcap}% it must be loaded after hyperref.
% Set up styles of URL: it should be placed after hyperref.
\urlstyle{same}
\addto\captionsenglish{\renewcommand{\contentsname}{Contents:}}

\usepackage{sphinxmessages}
\setcounter{tocdepth}{1}



\title{Learning Notes}
\date{2020 年 01 月 01 日}
\release{0.0.1}
\author{Liu Qiang}
\newcommand{\sphinxlogo}{\vbox{}}
\renewcommand{\releasename}{发布}
\makeindex
\begin{document}

\pagestyle{empty}
\sphinxmaketitle
\pagestyle{plain}
\sphinxtableofcontents
\pagestyle{normal}
\phantomsection\label{\detokenize{index::doc}}



\chapter{求职简历}
\label{\detokenize{resume:id1}}\label{\detokenize{resume::doc}}

\section{个人信息}
\label{\detokenize{resume:id2}}
\$a\textasciicircum{}\{2\}+b\textasciicircum{}\{2\}=c\textasciicircum{}\{2\}\$
\sphinxhyphen{} 刘强/男/1992
\sphinxhyphen{} 最高学历:硕士,课题研究方向:机器人运动规划
\sphinxhyphen{} 工作经验:1.5年
\sphinxhyphen{} CSDN博客:
\sphinxhyphen{} 个人主页:
\sphinxhyphen{} GitHub:
\sphinxhyphen{} 期望职位:机器人软件算法工程师
\sphinxhyphen{} 期望薪资:税前月薪15k\textasciitilde{}20k,特别喜欢的公司例外
\sphinxhyphen{} 期望城市:武汉

求职意向:机器人软件算法工程师


\section{实习经历}
\label{\detokenize{resume:id3}}
\sphinxstylestrong{武汉库柏特科技有限公司}

2018/10~2019/01:在**机器人与智能系统部**工作,作为主力参与公司的深度学习模型训练工具的开发,编程语言使用的是C++,主要用到的软件框架是Qt5和Caffe。


\section{工作经历}
\label{\detokenize{resume:id4}}
\sphinxstylestrong{武汉库柏特科技有限公司}

2019/06\textasciitilde{}2019/09:在**机器人部**工作,独立完成多款机器人的运动学封闭解算法的开发和接口封装,为公司的多个机器人项目(如**核清洗机器人**项目、{\color{red}\bfseries{}**}航天机器人**项目等)提供算法支持。开发机器人的轨迹规划和轨迹插补算法,为西安**某经颅磁刺激项目**提供算法支持。使用的编程语言有C++/Python,使用到的第三方库有Eigen3/Numpy/SymPy。其中,C++/Eigen3用于算法开发,Python/Numpy/SymPy用于算法的公式推导。

2019/10\textasciitilde{}2019/12:在**物流与拆码垛组**工作,期间作为主要成员参与公司的**某机器人重大专项**项目申报工作,作为主力参与**某电力行业机器人**项目的方案细化,参与公司的产品**机器人通用AI抓取平台**的研发工作。其中,双11和双12电商大促销期间,在武汉某大型化妆品物流中心维护公司的4台UR10机器人分捡工作站。参与UR/Aubo/Elite/Denso/机器人的驱动开发。

2020/01\textasciitilde{}2020/03:在**机器人操作系统**工作,主持完成公司产品的文档化工作,并部署在微软云上。完成公司的机器人选型库的开发。


\section{项目经验}
\label{\detokenize{resume:id7}}
2019/02~2019/05:参与导师的**某破损钻头激光修复**项目,项目中使用到了图像和点云数据处理,激光修复机器人的路径规划。


\section{专业技能}
\label{\detokenize{resume:id8}}
熟练使用C++及一些相关的库(如Qt5/STL/Eigen/Sophus/Ceres/OpenCV/PCL/VTK/OSG/Caffe)

熟练使用Python及一些相关的库(如Numpy/SymPy/Matplotlib/PyQt5/OpenCV/PyTorch)

熟练使用CMake/Ubuntu/Shell进行项目开发

熟练使用Git和Gitlab/Github进行代码开发和管理

熟练使用GCC/GDB进行编译和代码调试

熟练使用Clion/VSCode/QtCreator开发工具

有丰富的编写和整理开发文档经验。

熟悉机器人运动学和动力学知识(独立推导和实现多款机器人封闭解算法,实现三次/五次/梯形/S型/B样条轨迹插补,实现过机器人的牛顿\sphinxhyphen{}欧拉公式、拉格朗日公式)。

机器人控制

熟悉机器人的运动规划算法(OMPL/SBPL/TrajOpt/STOMP)

熟悉ROS/RobWork/MoveIt!/DART/Drake等机器人应用框架


\section{个人项目}
\label{\detokenize{resume:id9}}
数据流编程软件

图像处理软件

点云处理软件

运动仿真软件

深度学习软件(图像分类,目标检测,实例分割)

物体抓取软件

力控打磨软件

力控装配软件


\section{图书出版}
\label{\detokenize{resume:id10}}
《机器人软件算法工程师实战手册》


\section{教育经历}
\label{\detokenize{resume:id11}}\begin{itemize}
\item {} 
本科(2012/09~2016/07):就读于沈阳航空航天大学机械创新实验班。

\item {} 
硕士(2016/09~2019/06):就读于桂林电子科技大学机电工程学院。

\end{itemize}


\chapter{Indices and tables}
\label{\detokenize{index:indices-and-tables}}\begin{itemize}
\item {} 
\DUrole{xref,std,std-ref}{genindex}

\item {} 
\DUrole{xref,std,std-ref}{modindex}

\item {} 
\DUrole{xref,std,std-ref}{search}

\end{itemize}



\renewcommand{\indexname}{索引}
\printindex
\end{document}